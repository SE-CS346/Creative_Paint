%Copyright 2014 Jean-Philippe Eisenbarth
%This program is free software: you can 
%redistribute it and/or modify it under the terms of the GNU General Public 
%License as published by the Free Software Foundation, either version 3 of the 
%License, or (at your option) any later version.
%This program is distributed in the hope that it will be useful,but WITHOUT ANY 
%WARRANTY; without even the implied warranty of MERCHANTABILITY or FITNESS FOR A 
%PARTICULAR PURPOSE. See the GNU General Public License for more details.
%You should have received a copy of the GNU General Public License along with 
%this program.  If not, see <http://www.gnu.org/licenses/>.

%Based on the code of Yiannis Lazarides
%http://tex.stackexchange.com/questions/42602/software-requirements-specification-with-latex
%http://tex.stackexchange.com/users/963/yiannis-lazarides
%Also based on the template of Karl E. Wiegers
%http://www.se.rit.edu/~emad/teaching/slides/srs_template_sep14.pdf
%http://karlwiegers.com
\documentclass{scrreprt}
\usepackage{listings}
\usepackage{underscore}
\usepackage[bookmarks=true]{hyperref}
\usepackage[utf8]{inputenc}
\usepackage[english]{babel}
\usepackage{array}
\usepackage{enumitem}
\usepackage[table, dvipsnames]{xcolor}
\newcommand{\CL}{\fontfamily{cmtt}\selectfont}
\newcommand{\PL}{\fontfamily{phv}\selectfont}
\setcounter{tocdepth}{7}

\usepackage{graphicx}
\hypersetup{
    bookmarks=false,    % show bookmarks bar?
    pdftitle={Software Requirement Specification},    % title
    pdfauthor={Jean-Philippe Eisenbarth},                     % author
    pdfsubject={TeX and LaTeX},                        % subject of the document
    pdfkeywords={TeX, LaTeX, graphics, images}, % list of keywords
    linkcolor=blue,       % color of internal links
    citecolor=black,       % color of links to bibliography
    filecolor=black,        % color of file links
    urlcolor=purple,        % color of external links
}%
\def\myversion{1.0 }
\date{}
%\title
\usepackage{hyperref}
\begin{document}

\begin{flushleft}
    \rule{14.5cm}{5pt}\vskip1cm
    \begin{bfseries}
        \huge{\centerline{SOFTWARE REQUIREMENTS}}
        \huge{\centerline{SPECIFICATION}}
        \vspace{1.5cm}
        \LARGE \centerline{CS346: SOFTWARE ENGINEERING LABORATORY} 
        \vspace{2cm}
        \LARGE \centerline{Group 11} 
        \LARGE \centerline{Project 7: Paint Application}
        \vspace{2cm}
        \LARGE \centerline{Samay Varshney  180101097}
        \centerline{Siddhartha Jain  180101078}
        \centerline{Varhade Amey Anant  180101087}
        \centerline{Pulkit Changoiwala  180101093}
        \vspace{4cm}
        \LARGE \centerline{Department of Computer Science and Engineering}
        \centerline{Indian Institute of Technology Guwahati}
        \vspace{1.9cm}
    \end{bfseries}
\end{flushleft}

\tableofcontents

\chapter{Introduction}
This Software Requirement Specification (SRS) Document describes an
android/web application that provides children to draw, paint, and create their own art along with having fun. 

\section{Purpose}
The purpose of this document is to build an offline painting application to aid children to develop their mobility and creativity skills.

\section{Document Conventions}
The format specified by IEEE was followed while creating this document.
\begin{itemize}[itemsep=0.5pt]
    \item User: Person interacting with the application
    \item SRS: Software Requirements Specifications
    \item APK : Android Package
\end{itemize}

\section{Intended Audience and Reading Suggestions}
This project is a prototype of the paint application for children of five to nine years age. The intended audience for this SRS is Professor Samit Bhattacharya, who is in charge of the CS-346: Software Engineering course. 

\section{Project Scope}
The software that this SRS specifies is the Creative Paint (CP) Application. Its goal is to create a convenient and easy-to-use application for children, trying to have fun while drawing. Above all, we hope to provide a comfortable user experience.

\section{References}
IEEE, IEEE Standard 830-1998 IEEE Recommended Practice for Software Requirements Specifications, IEEE Computer Society, 1998.

\section{Overview}
The remaining part of the SRS contains:
\begin{itemize}[itemsep=0.5pt]
    \item The Overall Description and Functioning of the Software
    \item Specific Requirements:
        \begin{itemize}[itemsep=0.5pt]
            \item Functional - defining the fundamental actions that the software incorporates in accepting and processing the inputs and corresponding outputs.
            \item Non-Functional - software system attributes that are used
            to judge the operation of the system.
        \end{itemize}
\end{itemize}

\chapter{Overall Description}

\section{Product Perspective}
The product described in the document is a Paint App. It is a stand-alone Android application that can be launched after installing via a standard freely-distributed APK file.
The product is envisioned to give children of the age group 5yrs to 9yrs a platform to explore their creative skills. It aims to provide a user friendly experience to them. Equipped with two different modes: 1)Game Mode 2)Practice Mode, the paint app offers a wholesome experience to art lovers. 

\section{Product Functions}
Functions included in the final app will be as follows:
\begin{enumerate}[itemsep=0.5pt]
    \item Paint App Start Menu
    \item Practice Mode
    \item Game Mode
    \item My Work Mode
    \item Inspiration Mode
\end{enumerate}

\section{User Classes and Characteristics}
Information regarding users who can use the app:
    \begin{enumerate}[itemsep=0.5pt]
        \item Our app is targeted mainly for children between age range 5-9 years.
        \item Children who just wanted to play for fun of any age.
        \item Children who wanted to develop their art skills starting from basic levels. 
    \end{enumerate}
The intended users for the product will have the following characteristics:
    \begin{enumerate}[itemsep=0.5pt]
        \item Able to understand the functioning and operation of the software on a basic level.
        \item Able to afford and use a mobile app following the minimum hardware and software requirements.
        \item The user is able to understand English.
        \item The users of all ages can use our app.
    \end{enumerate}

\section{Operating Environment}
Our app will work on an android device with at least 512 MB RAM.

\section{Assumptions and Dependencies}
Performance of the app will depend on hardware configuration of device and operating system in use.
Assumptions:
\begin{enumerate}[itemsep=0.5pt]
    \item Device uses Android 5 or higher / API level 24 or higher.
    \item Touch screen
\end{enumerate}

\chapter{Functional Requirements}
\section{Paint App Start}
\textbf{Input:} App Icon \newline
\textbf{Output:} List of Options
\vspace{1mm}\newline
\textbf{Description:} \newline 
When user starts the app, the main menu appears on the screen. Main menu consists of options like practice mode, game mode, my work and motivation.

\subsection{Practice Mode}
\textbf{Input:} Practice Mode \newline
\textbf{Output:} List of Modes
\vspace{1mm}\newline
\textbf{Description:} \newline 
Practice mode offers two options:  Free Style and Image Practicing.

\subsubsection{3.1.1.1 Freestyle Mode}
\textbf{Input:} Freestyle Mode \newline
\textbf{Output:} Drawing boards Layouts
\vspace{1mm}\newline
\textbf{Description:} \newline 
User is prompted to select a drawing board from different choices to practice his/her skills.

\paragraph{3.1.1.1.1 Drawing Board Layout}
\hfill \vspace{2.5mm} \break 
\textbf{Input:} Drawing Board Layout \newline
\textbf{Output:} Drawing board and Toolbox
\vspace{1mm}\newline
\textbf{Description:} \newline 
User selects a board from given alternatives. A toolbox along with white board appears.

\paragraph{3.1.1.1.1.1 Toolbox}
\hfill \vspace{2.5mm} \break 
\textbf{Description:} \newline
User has various tools which
he/she can use it to draw or paint. It appears along with the drawing board layout. It offers the following tools for the drawing.

\paragraph{3.1.1.1.1.1.1 Pencil Tool}
\hfill \vspace{2.5mm} \break 
\textbf{Input:} Size and Colour \newline
\textbf{Output:} Pencil
\vspace{1mm}\newline
\textbf{Description:} \newline 
User can choose a pencil with a specified size and colour.

\paragraph{3.1.1.1.1.1.2 Select Eraser}
\hfill \vspace{2.5mm} \break 
\textbf{Input:} Size \newline
\textbf{Output:} Eraser
\vspace{1mm}\newline
\textbf{Description:} \newline 
User can choose a pencil with a specified size and colour.

\paragraph{3.1.1.1.1.1.3 Brush Tool}
\hfill \vspace{2.5mm} \break 
\textbf{Input:} Size and Colour \newline
\textbf{Output:} Brush
\vspace{1mm}\newline
\textbf{Description:} \newline 
User can choose brush to draw an image. He/She has the option to choose a size of the brush.

\paragraph{3.1.1.1.1.1.4 Colour Fill}
\hfill \vspace{2.5mm} \break 
\textbf{Input:} Colour and Image \newline
\textbf{Output:} Coloured Image
\vspace{1mm}\newline
\textbf{Description:} \newline 
User can fill colour in an image.He/she has to choose a color and image in which they want to fill the colour.

\paragraph{3.1.1.1.1.1.5 Select Shape}
\hfill \vspace{2.5mm} \break 
\textbf{Input:} Size and Colour \newline
\textbf{Output:} Pencil
\vspace{1mm}\newline
\textbf{Description:} \newline 
User will be given different types of the shape to choose from. He/She will select the shape and it will appear on the board.

\paragraph{3.1.1.1.1.1.6 Undo}
\hfill \vspace{2.5mm} \break 
\textbf{Input:} Type \newline
\textbf{Output:} Shape
\vspace{1mm}\newline
\textbf{Description:} \newline 
User will use this option to revert back or undone the recent work done.

\paragraph{3.1.1.1.1.1.7 Save Image}
\hfill \vspace{2.5mm} \break 
\textbf{Input:} Size and Colour \newline
\textbf{Output:} Pencil
\vspace{1mm}\newline
\textbf{Description:} \newline 
Users will be able to save his work so that they can see it in future.

\paragraph{3.1.1.1.1.1.8 Share Image}
\hfill \vspace{2.5mm} \break 
\textbf{Input:} Size and Colour \newline
\textbf{Output:} Pencil
\vspace{1mm}\newline
\textbf{Description:} \newline 
Users will be able to share their artwork on social media.

\subsubsection{3.1.1.2 Image Practicing Mode}
\textbf{Input:} Image Practicing Mode \newline
\textbf{Output:} Drawing board Layouts
\vspace{1mm}\newline
\textbf{Description:} \newline 
User is prompted to select a drawing board from different choices to practice his/her skills.

\paragraph{3.1.1.2.1 Drawing Board Layout}
\hfill \vspace{2.5mm} \break 
\textbf{Input:} Drawing Board Layout \newline
\textbf{Output:} Drawing board and Toolbox
\vspace{1mm}\newline
\textbf{Description:} \newline 
User selects a board from given alternatives. A toolbox along with white board appears.

\paragraph{3.1.1.2.1.1 Image Selection}
\hfill \vspace{2.5mm} \break 
\textbf{Input:} Image \newline
\textbf{Output:}  Drawing board, Toolbox and Image
\vspace{1mm}\newline
\textbf{Description:} \newline 
User selects an image from given alternatives. A toolbox and white board along with a selected image appears.

\paragraph{3.1.1.2.1.2 Toolbox}
\hfill \vspace{2.5mm} \break 
\textbf{Description:} \newline
User has various tools which
he/she can use it to draw or paint. It appears along with the drawing board layout. It offers the following tools for the drawing.
\paragraph{3.1.1.2.1.1.1 Pencil Tool}
\hfill \vspace{2.5mm} \break 
\textbf{Input:} Size and Colour \newline
\textbf{Output:} Pencil
\vspace{1mm}\newline
\textbf{Description:} \newline 
User can choose a pencil with a specified size and colour.

\paragraph{3.1.1.2.1.1.2 Select Eraser}
\hfill \vspace{2.5mm} \break 
\textbf{Input:} Size \newline
\textbf{Output:} Eraser
\vspace{1mm}\newline
\textbf{Description:} \newline 
User can choose a pencil with a specified size and colour.

\paragraph{3.1.1.2.1.1.3 Brush Tool}
\hfill \vspace{2.5mm} \break 
\textbf{Input:} Size and Colour \newline
\textbf{Output:} Brush
\vspace{1mm}\newline
\textbf{Description:} \newline 
User can choose brush to draw an image. He/She has the option to choose a size of the brush.

\paragraph{3.1.1.2.1.1.4 Colour Fill}
\hfill \vspace{2.5mm} \break 
\textbf{Input:} Colour and Image \newline
\textbf{Output:} Coloured Image
\vspace{1mm}\newline
\textbf{Description:} \newline 
User can fill colour in an image.He/she has to choose a color and image in which they want to fill the colour.

\paragraph{3.1.1.2.1.1.5 Select Shape}
\hfill \vspace{2.5mm} \break 
\textbf{Input:} Size and Colour \newline
\textbf{Output:} Pencil
\vspace{1mm}\newline
\textbf{Description:} \newline 
User will be given different types of the shape to choose from. He/She will select the shape and it will appear on the board.

\paragraph{3.1.1.2.1.1.6 Undo}
\hfill \vspace{2.5mm} \break 
\textbf{Input:} Type \newline
\textbf{Output:} Shape
\vspace{1mm}\newline
\textbf{Description:} \newline 
User will use this option to revert back or undone the recent work done.

\paragraph{3.1.1.2.1.1.7 Save Image}
\hfill \vspace{2.5mm} \break 
\textbf{Input:} Size and Colour \newline
\textbf{Output:} Pencil
\vspace{1mm}\newline
\textbf{Description:} \newline 
Users will be able to save his work so that they can see it in future.

\paragraph{3.1.1.2.1.1.8 Share Image}
\hfill \vspace{2.5mm} \break 
\textbf{Input:} Size and Colour \newline
\textbf{Output:} Pencil
\vspace{1mm}\newline
\textbf{Description:} \newline
Users will be able to share their artwork on social media.

\subsubsection{Game Mode}
\textbf{Input:} Game Mode \newline
\textbf{Output:} List of Levels
\vspace{1mm}\newline
\textbf{Description:} \newline 
Game mode starts. It is the game zone of the app where the user has many levels of the game to take part. The user can choose the current unfinished level or from the levels which the user has finished.

\subsubsection{3.1.2.1 Select a Level}
\textbf{Input:} Level \newline
\textbf{Output:} Drawing board Layouts
\vspace{1mm}\newline
\textbf{Description:} \newline 
User is prompted to select a drawing board from different choices to practice his/her skills.

\paragraph{3.1.2.1.1 Drawing Board Layout}
\hfill \vspace{2.5mm} \break 
\textbf{Input:} Drawing Board Layout \newline
\textbf{Output:} Drawing board and Toolbox
\vspace{1mm}\newline
\textbf{Description:} \newline 
User selects a board from given alternatives. A toolbox along with white board appears.

\paragraph{3.1.2.1.1.1 Toolbox}
\hfill \vspace{2.5mm} \break 
\textbf{Description:} \newline
User has various tools which
he/she can use it to draw or paint. It appears along with the drawing board layout. It offers the following tools for the drawing.





\paragraph{3.1.2.1.1.1.1 Pencil Tool}
\hfill \vspace{2.5mm} \break 
\textbf{Input:} Size and Colour \newline
\textbf{Output:} Pencil
\vspace{1mm}\newline
\textbf{Description:} \newline 
User can choose a pencil with a specified size and colour.

\paragraph{3.1.2.1.1.1.2 Select Eraser}
\hfill \vspace{2.5mm} \break 
\textbf{Input:} Size \newline
\textbf{Output:} Eraser
\vspace{1mm}\newline
\textbf{Description:} \newline 
User can choose a pencil with a specified size and colour.

\paragraph{3.1.2.1.1.1.3 Brush Tool}
\hfill \vspace{2.5mm} \break 
\textbf{Input:} Size and Colour \newline
\textbf{Output:} Brush
\vspace{1mm}\newline
\textbf{Description:} \newline 
User can choose brush to draw an image. He/She has the option to choose a size of the brush.

\paragraph{3.1.2.1.1.1.4 Colour Fill}
\hfill \vspace{2.5mm} \break 
\textbf{Input:} Colour and Image \newline
\textbf{Output:} Coloured Image
\vspace{1mm}\newline
\textbf{Description:} \newline 
User can fill colour in an image.He/she has to choose a color and image in which they want to fill the colour.

\paragraph{3.1.2.1.1.1.5 Select Shape}
\hfill \vspace{2.5mm} \break 
\textbf{Input:} Size and Colour \newline
\textbf{Output:} Pencil
\vspace{1mm}\newline
\textbf{Description:} \newline 
User will be given different types of the shape to choose from. He/She will select the shape and it will appear on the board.

\paragraph{3.1.2.1.1.1.6 Undo}
\hfill \vspace{2.5mm} \break 
\textbf{Input:} Type \newline
\textbf{Output:} Shape
\vspace{1mm}\newline
\textbf{Description:} \newline 
User will use this option to revert back or undone the recent work done.

\paragraph{3.1.2.1.1.1.7 Save Image}
\hfill \vspace{2.5mm} \break 
\textbf{Input:} Size and Colour \newline
\textbf{Output:} Pencil
\vspace{1mm}\newline
\textbf{Description:} \newline 
Users will be able to save his work so that they can see it in future.

\paragraph{3.1.2.1.1.1.8 Share Image}
\hfill \vspace{2.5mm} \break 
\textbf{Input:} Size and Colour \newline
\textbf{Output:} Pencil
\vspace{1mm}\newline
\textbf{Description:} \newline 
Users will be able to share their artwork on social media.

\paragraph{3.1.2.1.1.1.8 Submit}
\hfill \vspace{2.5mm} \break 
\textbf{Input:} Image \newline
\textbf{Output:} Rewards
\vspace{1mm}\newline
\textbf{Description:} \newline 
Image is processed and according to percentage of level completion, rewards are given.

\subsection{My Work}
\textbf{Input:} My Work Mode \newline
\textbf{Output:} Images
\vspace{1mm}\newline
\textbf{Description:} \newline 
In this section the user can see all the work he/she has saved in the past. All of his/her past paintings are present in this section. 

\subsubsection{3.1.3.1 Delete Images}
\textbf{Input:} Cancel \newline
\textbf{Output:} "My Work" Section
\vspace{1mm}\newline
\textbf{Description:} \newline 
On clicking the delete option a dialogue box appears. It has two operation types which user can select: 1) Cancel  2) Delete.

\paragraph{3.1.3.1.1 Cancel Operation}
\hfill \vspace{2.5mm} Cancel\break 
\textbf{Input:} "My Work" Section \newline
\textbf{Output:} Drawing board and Toolbox
\vspace{1mm}\newline
\textbf{Description:} \newline 
User will go one step back to My Work section where all his previous work is displayed hence cancelling the deletion.

\paragraph{3.1.3.1.2 Delete Operation}
\hfill \vspace{2.5mm} \break 
\textbf{Input:} Drawing Board Layout \newline
\textbf{Output:} Drawing board and Toolbox
\vspace{1mm}\newline
\textbf{Description:} \newline 
If user selects the delete operation, image is being deleted and user will go one step back to My Work section

\subsubsection{3.1.3.2 Image Menu}
\textbf{Input:} Image \newline
\textbf{Output:} List of operations
\vspace{1mm}\newline
\textbf{Description:} \newline 
On long pressing an image, options for renaming, deletion and sharing of that particular image appears.

\paragraph{3.1.3.2.1 Rename Image}
\hfill \vspace{2.5mm} \break 
\textbf{Input:} Image and Name \newline
\textbf{Output:} Renamed Image
\vspace{1mm}\newline
\textbf{Description:} \newline 
After entering the preferred name for the image by the user, image name changed to that name for easier future references.

\paragraph{3.1.3.2.2 Delete Image}
\hfill \vspace{2.5mm} \break 
\textbf{Input:} Image \newline
\textbf{Output:} Image Deleted
\vspace{1mm}\newline
\textbf{Description:} \newline 
On clicking the delete button of the image, a dialogue box appears to confirm deletion or not.


\paragraph{3.1.3.2.3 Share Image}
\hfill \vspace{2.5mm} \break 
\textbf{Input:} Image \newline
\textbf{Output:} Image shared
\vspace{1mm}\newline
\textbf{Description:} \newline 
On clicking the share button of the image, different options (whatsapp, facebook, instagram) to share that particular image appears.


\subsection{Inspiration}
\textbf{Input:} Inspiration Mode \newline
\textbf{Output:} Paintings
\vspace{1mm}\newline
\textbf{Description:} \newline 
This section shows different drawable images to give an inspiration to the child and motivate him/her to draw. He/She can see different drawings by scrolling left and right. Images with drawing and its title appears.

\chapter{External Interface Requirements}
On clicking the share button of the image, different options (whatsapp, facebook, instagram) to share that particular image appears.

\section{User Interfaces}
\textcolor{red}{Describe the logical characteristics of each interface between the software product and the users. This may include sample screen images, any GUI standards or product family style guides that are to be followed, screen layout constraints, standard buttons and functions (e.g., help) that will appear on every screen, keyboard shortcuts, error message display standards, and so on.  
Define the software components for which a user interface is needed. Details of the user interface design should be documented in a separate user interface specification.}

\section{Hardware Interfaces}
The application has no designated hardware so there are no direct hardware interfaces.

\section{Software Interfaces}
The game will run on all devices supporting Android 5.0(Lollipop) API 21 and above.

\section{Communications Interfaces}
Only communication interface is the internet. It allows the user to share his/her art on social media.

\chapter{Nonfunctional Requirements}

\section{Performance Requirements}
512MB RAM: Better and more RAM will help in loading the pages instantly.

\section{Availability}
The system will be available for use whenever the user deems necessary 24/7. The system shall allow users to restart the application after failure of the app with the earlier user saved works and settings.

\section{Maintainability}
The system will be updatable from software patches available through the App Store. Updates can be downloaded through the standard Android interface. Any discrepancies will be addressable by any developer as the coding will be done according to the coding standards of IEEE.

\section{Portability}
Since the app is developed using the android studio, it will run on any android device and windows supporting android applications.
The paint software can be used on any Android phone satisfying the minimum hardware/software dependencies as specified in this SRS document previously. Installation of this application can be done through the standard Android File Manager, and this application can be shared through an APK file between devices.

\section{Reliability}

\begin{itemize}[itemsep=0.5pt]
    \item The software will be able to run 99\% of the time when launched.
    \item There is a potential for errors relating to the state of the operating system that could prevent the game from launching (for example not enough resources available, etc.). The chance of such an occurrence is at most 1\%.
    \item The application will stop running if sent to run in the background, or the phone is accidentally powered off.
    \item The system will not be prone to errors caused by unexpected input.
\end{itemize}

\section{Usability}

\subsection{Contextual Inquiry (CI)}
\vspace{-2pt}
One member from our team observed his 7 year old relative and another member did for his 9 year old neighbour. We set up our contextual enquiry in the following format. One of the contextual inquiry can be considered as active and another one as passive.
\vspace{-2pt}
\begin{enumerate}[itemsep=0.5pt]
    \item \textbf{Plan} \\ 
     One of the sessions was physical and the other one was a recorded one.
     The goals of the session are planned as below:
    \begin{enumerate}[itemsep=0.5pt]
        \item How do children learn to draw ?
        \item Main difficulty faced in drawing ?
        \item Preferred way of learning art by child ?
        \item How interested or enthusiastic is the child about art ?
    \end{enumerate}
    
    \item \textbf{Initiate} \\
    We took permission from our relatives regarding taking the feedback from their children. We had set up sessions of about 90 minutes each where the children did not formally know about it. The overall environment of the child was not disturbed and the interaction was casual yet professional.
    
    \item \textbf{Execute}
    \begin{enumerate}[itemsep=0.5pt]
        \item Recording the observations
        \item Asking relevant queries and clarifications
    \end{enumerate}
    \vspace{3pt}
    
    \begin{tabular}{|| m{0.25\textwidth } | m{0.35\textwidth} | m{0.35\textwidth}||}
    \hline \rowcolor{lightgray} \hline
    \textbf{Patterns/Areas} & \textbf{Child A} & \textbf{Child B}\\ [0.5ex] 
    \hline \hline
    \textbf{Context of use} & Primarily for school work, not interested in art in general. & Keen interest in art, uses this as a leisure and hobby activity. \\ \hline
    \textbf{Pain points and issues in physical drawing} & Not getting hooked to it, difficulty in learning because of handling of material  & Difficulty in handling material \\ \hline
    \textbf{Imagination and creativity} & Good at imagining & He faces difficulty in drawing completely on his own \\ \hline
    \textbf{Interest and Enthusiasm} & Not much enthusiasm in drawing as such in traditional method & Very much excited and enthusiastic about new prospects. \\ \hline
    \textbf{Parents interference and support} & Parents do not want to give exposure to smartphones at this age. & Parents do not have time to monitor whether their child is learning properly. \\ \hline
    \textbf{Software, Applications} & Uses smartphones under parents guidance. & Uses various applications and is quite familiar with digital entertainment as well as learning. \\ \hline
    \end{tabular}

    \item \textbf{Close}
            We thanked them for their cooperation and giving their ideas and suggestions regarding art.

    \item \textbf{Reflect}
            Analyzing the data generated through Affinity diagram: 
    \begin{figure}[!h]
        \centering
        \includegraphics[width=\textwidth]{3.png}
        \caption{Affinity Diagram}
        \label{Reflect}
    \end{figure}

\end{enumerate}

\end{document}